%
%  objectives
%
%  Created by Mike Helmick on 2007-05-09.
%  Copyright (c) 2007 Mike Helmick. All rights reserved.
%
\documentclass[10pt]{article}

% Use utf-8 encoding for foreign characters
\usepackage[utf8]{inputenc}

% Setup for fullpage use
\usepackage{fullpage}
\usepackage{url}

% Uncomment some of the following if you use the features
%
% Running Headers and footers
%\usepackage{fancyheadings}

% Multipart figures
%\usepackage{subfigure}

% More symbols
%\usepackage{amsmath}
%\usepackage{amssymb}
%\usepackage{latexsym}

% Surround parts of graphics with box
\usepackage{boxedminipage}

% Package for including code in the document
\usepackage{listings}

% If you want to generate a toc for each chapter (use with book)
\usepackage{minitoc}


\newif\ifpdf
\ifx\pdfoutput\undefined
\pdffalse % we are not running PDFLaTeX
\else
\pdfoutput=1 % we are running PDFLaTeX
\pdftrue
\fi

\ifpdf
\usepackage[pdftex]{graphicx}
\else
\usepackage{graphicx}
\fi
\title{CSA174 - Fall 2007 \\ Learning Objectives \& Outcomes}
\author{ Mike Helmick, PhD \thanks{mike.helmick@muohio.edu - https://my.csi.muohio.edu} \\
Visiting Assistant Professor \\
Department of Computer Science and Systems Analysis \\
Miami University \\ Oxford, Ohio 45056 }

\date{}

\begin{document}

\ifpdf
\DeclareGraphicsExtensions{.pdf, .jpg, .tif}
\else
\DeclareGraphicsExtensions{.eps, .jpg}
\fi

\maketitle

\section{Learning Objectives \& Outcomes}

The standard syllabus for this class is available at \newline
\url{http://csa.muohio.edu/courseDescriptions/174.html}

\begin{enumerate}
	\item To provide students with experience developing computer programs in a modern programming language.
	\begin{enumerate}
		\item The student can identify and describe the different representations of a program and the compilation process.
		\item The student can create a project and execute a program in a modern development environment.
		\item The student can debug their own and other's program code.
		\item The student can test their code for correct behavior. 
		
	\end{enumerate}
	\item To provide student with a foundation for problem solving, and expressing solutions as computer programs.
	\begin{enumerate}
		\item The student can define and describe the role of algorithms.
		\item The student can create a program solving a problem that is sequential in nature.
		\item The student can create a program solving a problem using conditional logic.
		\item The student can create a program solving a problem that uses repetition.
		\item The student can create a program that uses combinations of sequential, conditional, and repetitive logic.
		\item The student can represent program elements using primitive data types.
		\item The student can develop programs that interact with the user using text based input and output.
		\item The student can create and use custom data types.
		\item The student can design custom data types that solve specified problems.
	\end{enumerate}
	\item To provide the student with experience and manipulating multi-dimensional data.
	\begin{enumerate}
		\item The student can create a program that uses both one and two-dimensional arrays of both primitive and complex data types.
		\item The student can implement algorithms for searching two-dimensional data.
		\item The student can implement algorithms for sorting two-dimensional data.
	\end{enumerate}
\end{enumerate}

\bibliography{}
\end{document}
