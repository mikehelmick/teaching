\documentclass[11pt,answers]{exam}
\usepackage{listings}
\usepackage{pdfsync}

%
%  Created by Mike Helmick on 2006-09-13.
%  Copyright (c) 2006 Mike Helmick. All rights reserved.
%
%

\newif\ifpdf
\ifx\pdfoutput\undefined
\pdffalse % we are not running PDFLaTeX
\else
\pdfoutput=1 % we are running PDFLaTeX
\pdftrue
\fi

\ifpdf
\usepackage{subfigure}
\usepackage[pdftex]{graphicx}
\else
\usepackage{graphicx}
\fi

% exam settings
%\boxedpoints
%\pointsinmargin
%\printanswers 
\noprintanswers

\usepackage{color} 
\definecolor{SolutionColor}{rgb}{0.8,0.9,0.9} 
\shadedsolutions 


%
%  Update these values for running headers
%
\firstpageheader{\bf\Large CSA174}{\bf\Large FINAL EXAM - Part 1}{\bf\Large
  2007-12-11 }
\runningheader{CSA 174 FINAL EXAM}{Miami University}{Part 1, Form 1}
\addpoints

\begin{document}

\begin{center} 
  \fbox{\fbox{\parbox{5.5in}{\centering
    CSA174 - Fall 2007 - Final Exam \newline
	Miami University \newline
	\newline
	There are 105 possible points on this exam.\newline
	Your grade will be calculated out of 100 points.\newline
 	In this part there are \numquestions\  questions for a total of  \numpoints\ points.
}}}
\end{center} 

% setup standard options for the including code fragments
\lstset{language=Java,numbers=left, numberstyle=\tiny, stepnumber=1, numbersep=5pt, showstringspaces=true}

\vspace{0.1in} 

\section*{Instructions}
\begin{itemize}
 \item This exam is being administered in two parts - the scantron type answer section and programming questions
 \item This exam is closed book - you may have 1 piece of 8.5” x 11” paper that you have prepared in advance.
 \item No electronic devices may be used during the exam: this includes calculators, iPods, PDAs, and cellular phones. 
 \item You have 120 minutes (2 hours and 0 minutes) to complete BOTH sections of the exam. 
 \item Do not mark answer on this testing booklet - they will not be graded.   You may use the space in this booklet for computations and working through problems.
\end{itemize}

\section*{On your Miami Testing form}
\begin{enumerate}
	\item {\bf Write your name, course, section, instructor AND bubble in your UNIQUEID (computer login ID) on the right side of the page}
	\item {\bf Bubble in your form number (yes, there are different tests)}
	\item {\bf Make sure you completely fill in the appropriate circle and completely erase all stray marks}
\end{enumerate}	

\section*{Form Number}
\begin{center}
	\fbox{\fbox{\parbox{5.5in}{\centering
	{\huge{{\bf TEST FORM NUMBER 1 (ONE)}}}
	}}}
\end{center}

\newpage

% Questions start here:
\begin{questions}

\section*{True False}
\uplevel{Fill in either “A” (True) or “B” (False) on your answer sheet}

\question[1] True or False: The program that converts source code into Java byte-code instructions is referred to as the java compiler.

\question[1] True or False: The following code will produce a syntax error: \newline
\begin{tt}
double x = 3; \newline
int y = x; \newline
System.out.println(x + y);
\end{tt}

\question[1] True or False: In the code below, bodyTemperature is initialized to the value null. \newline
\begin{tt}
double[] bodyTemperature;
\end{tt}

\question[1] True or False:In the array defined below, each of the elements in the array is initialized to a value of null: \newline
\begin{tt}
char[] c = new char[15];
\end{tt}

\question[1] True or False:   A two-dimensional array is an array of one-dimensional arrays in which the lengths of each of the one-dimensional arrays must be equal.

\question[1] True or False:  In a class, if you define a constructor for a class, but you don't define a no-argument (default) constructor, then a no-argument constructor will be automatically generated for you.

\question[1] True of False:  In a class, if you do not define any constructors, then a default, no-argument constructor will be automatically generated for you.

\question[1] True or False:  If the class Book has an instance variable {\tt isbn} of type {\tt int}, and a static variable of type {\tt int}, and you create 100 objects of type {\tt Book}, this will result in 100 instance variables, but only one static variable being allocated in memory.

\question[1] True or False:  Strings have a {\it public accessor method} named length() and arrays have a {\it public instance variable} named length.

\question[1] True or False:  The main method must always take in a single parameter of type {\tt String[]}.

\question[1] True or False:  If your program contains a syntax error, it will compile and run, but might crash while running.

\question[1] True or False: int, float, boolean, and String are examples of Java's primitive types.

\question[1] True or False: The Java {\it for} loop always executes the loop body at least one time.

\question[1] True or False: The and {\tt \&\&} and or {\tt ||} operators can only be applied to boolean expressions that appear in {\tt if} statements.

\question[1] True or False: In java, {\tt 13/4} is exactly equal to {\tt 3}.

\newpage
\section*{Multiple Choice}
\uplevel{Select the {\bf BEST} answer.  All questions have exactly 1 answer.}

\uplevel{Questions 16 through 20 use the following declarations: \newline
  \begin{tt}
    \hspace{.5in} int[][] a = new int[5][10];\newline
    \hspace{.5in} Airplane[][] b = new Airplane[5][10];\newline
    \hspace{.5in} boolean[] c = new boolean[2];	\newline
  \end{tt}
}

\question[1] What is the value of {\tt a[5].length}? \newline
\begin{oneparchoices}
 \choice 0 \choice 5 \choice 10 \choice error	
\end{oneparchoices}

\question[1] What is the value of {\tt a.length}? \newline
\begin{oneparchoices}
 \choice 0 \choice 5 \choice 10 \choice error	
\end{oneparchoices}

\question[1] What is the value of {\tt a[2][3]}? \newline
\begin{oneparchoices}
 \choice -1 \choice 0 \choice 1 \choice null	
\end{oneparchoices}

\question[1] What is the value of {\tt b[2][3]}? \newline
\begin{oneparchoices}
 \choice -1 \choice 0 \choice null \choice Some instance of the Airplane class	
\end{oneparchoices}

\question[1] What is the value of {\tt c[2]}? \newline
\begin{oneparchoices}
 \choice true \choice false \choice error
\end{oneparchoices}

\question[1] Which one of the following lines of code contains a syntax error? \newline
\begin{oneparchoices}
  \choice {\tt double a = 3.0;}
  \choice {\tt double b = 3;}
  \choice {\tt int c = 3.0;}
  \choice {\tt int d = 3;}	
\end{oneparchoices}

\question[1] When you use the assignment operator with variables of a class type, \newline you are assigning a: \newline
\begin{oneparchoices}
	\choice local variable \choice value \choice primitive type \choice reference
\end{oneparchoices}

\question[1] When you use the assignment operator with variables of any primitive type, \newline you are assigning a: \newline
\begin{oneparchoices}
	\choice local variable \choice value \choice primitive type \choice reference
\end{oneparchoices}

\question[1] Default constructors have \makebox[1in]{\hrulefill} parameter(s). \newline
\begin{oneparchoices}
	\choice exactly 0 \choice at least 1 \choice exactly 1 \choice at most 1
\end{oneparchoices}

\question[1] What happens if a program uses an array index that is out of bounds?
\begin{choices}
	\choice The compiler will give an error message, and not compile the program
	\choice The compiler will give a warning, but will still compile the program
	\choice The program will compile, but Java will given an error message when the program runs
	\choice The program will compile and run with no error messages, but you might see incorrect results
\end{choices}

\newpage

\question[1] How many times will this loop body execute? \newline
\begin{lstlisting}
int count = 50; 
while ( count >= 0 ) {
  System.out.println(count);
  count = count + 1;
}	
\end{lstlisting} 
\begin{oneparchoices}
	\choice 0 \choice 1 \choice 49 \choice 50 \choice more than 50
\end{oneparchoices}

\uplevel{Questions 27 through 30 use the following code segment.}
\begin{lstlisting}
int x = 5;
int y = 10;
int z = 0;
int w = 42;
for( int i = 0; i < x; i++ ) {
  for( int j = 0; j < y; j++ ) {
    int w = x;
    z = x * y;	
  }
}	
System.out.println( x + " " + y + " " + w + " " + z );
\end{lstlisting}

\question[1] When the code completes, what is the value of the variable {\tt x}? \newline
\begin{oneparchoices}
	\choice 0 \choice 5 \choice 10 \choice 50 \choice error
\end{oneparchoices}

\question[1] When the code completes, what is the value of the variable {\tt y}? \newline
\begin{oneparchoices}
	\choice 0 \choice 5 \choice 10 \choice 50 \choice error
\end{oneparchoices}

\question[1] When the code completes, what is the value of the variable {\tt w}? \newline
\begin{oneparchoices}
	\choice 42 \choice 5 \choice 10 \choice 50 \choice error
\end{oneparchoices}

\question[1] When the code completes, what is the value of the variable {\tt z}? \newline
\begin{oneparchoices}
	\choice 49 \choice 36 \choice 10 \choice 50 \choice 14
\end{oneparchoices}

\newpage
\section*{Code Analysis}

\uplevel{Use this code to answer questions 31 through 43.  It is probably best to calculate all positions for the arrays first, then answer the questions.}
\begin{lstlisting}
public static void doubleIt( int[] a, int v ) {
  v *= 2;
  for( int i = 0; i < a.length; i++ ) {
    a[i] = v;	
  }	
}	

public static void doubleIt( int[] a ) {
  for( int i = 0; i < a.length; i++ ) {
    a[i] *= 2;	 
  }
}

public static void doubleIt( int a ) {
  a *= 2;	
}

public static void main( String[] args ) {
  int[] a = { 1, 2, 3, 4 };
  int[] b = { 10, 20, 30, 40 };
  int[] c = { 50, 100, 150, 200, 250 };

  doubleIt( a ); 
  doubleIt( b[2] );
  doubleIt( c, c[0] );	
  // DETERMINE VALUES AT THIS POINT
}
\end{lstlisting}

\question[1] What is the value of {\tt a[0]} at {\bf line 26}? \newline
\begin{oneparchoices}
	\choice 0 \choice 1 \choice 2 \choice 4
\end{oneparchoices}
\question[1] What is the value of {\tt a[1]} at {\bf line 26}? \newline
\begin{oneparchoices}
	\choice 2 \choice 4 \choice 0 \choice 8
\end{oneparchoices}
\question[1] What is the value of {\tt a[2]} at {\bf line 26}? \newline
\begin{oneparchoices}
	\choice 2 \choice 4 \choice 6 \choice 8
\end{oneparchoices}
\question[1] What is the value of {\tt a[3]} at {\bf line 26}? \newline
\begin{oneparchoices}
	\choice 2 \choice 8 \choice 1 \choice 4
\end{oneparchoices}

\newpage

\question[1] What is the value of {\tt b[0]} at {\bf line 26}? \newline
\begin{oneparchoices}
	\choice 10 \choice 20 \choice 30 \choice 40 \choice error
\end{oneparchoices}
\question[1] What is the value of {\tt b[1]} at {\bf line 26}? \newline
\begin{oneparchoices}
	\choice 10 \choice 20 \choice 30 \choice 40 \choice error
\end{oneparchoices}
\question[1] What is the value of {\tt b[2]} at {\bf line 26}? \newline
\begin{oneparchoices}
	\choice 10 \choice 20 \choice 30 \choice 40 \choice error
\end{oneparchoices}
\question[1] What is the value of {\tt b[3]} at {\bf line 26}? \newline
\begin{oneparchoices}
	\choice 10 \choice 20 \choice 30 \choice 40 \choice error
\end{oneparchoices}

\makebox[4in]{\hrulefill}

\question[1] What is the value of {\tt c[0]} at {\bf line 26}? \newline
\begin{oneparchoices}
	\choice 50 \choice 100 \choice 150 \choice 200 \choice 250
\end{oneparchoices}
\question[1] What is the value of {\tt c[1]} at {\bf line 26}? \newline
\begin{oneparchoices}
	\choice 50 \choice 100 \choice 150 \choice 200 \choice 250
\end{oneparchoices}
\question[1] What is the value of {\tt c[2]} at {\bf line 26}? \newline
\begin{oneparchoices}
	\choice 50 \choice 100 \choice 150 \choice 200 \choice 250
\end{oneparchoices}
\question[1] What is the value of {\tt c[3]} at {\bf line 26}? \newline
\begin{oneparchoices}
	\choice 50 \choice 100 \choice 150 \choice 200 \choice 250
\end{oneparchoices}
\question[1] What is the value of {\tt c[4]} at {\bf line 26}? \newline
\begin{oneparchoices}
	\choice 50 \choice 100 \choice 150 \choice 200 \choice 250
\end{oneparchoices}


\makebox[4in]{\hrulefill}

\question[1] Identify the array declaration/initialization that is illegal (i.e., would be flagged by the compiler as a compile-time error).
\begin{choices}
	\choice {\tt int [] a = new int [10]; }
	\choice {\tt double [][] d = new double [10][]; }
	\choice {\tt char [][][] g = new char [5][5][5]; }
	\choice {\tt boolean [][] b = new boolean [4]; }
\end{choices}

\question[1] After the first pass of a selection sort, the array {\tt [ 4, 2, 5, 3, 1 ] } will be:
\begin{choices}
	\choice {\tt [ 4, 2, 1, 3, 5 ] }
	\choice {\tt [ 1, 2, 3, 4, 5 ] }
	\choice {\tt [ 1, 2, 5, 3, 4 ] } 
	\choice {\tt [ 4, 1, 2, 4, 5 ] }
\end{choices}

\newpage

\question[1] What is the result of this code on line 3?
\begin{lstlisting}
	String x;
	String y = "x";
	System.out.println( x.indexOf( y.charAt(0) ) );
\end{lstlisting}
\begin{oneparchoices}
	\choice 1
	\choice "x"
	\choice nothing
	\choice null pointer exception
\end{oneparchoices}

\question[1] What is the value of {\tt X} after this code executes?
\begin{lstlisting}
	String x = "HeLlO";
	x.toLowerCase();
\end{lstlisting}
\begin{oneparchoices}
	\choice "HeLlO" \choice "HELLO" \choice "hello" \choice "Hello"
\end{oneparchoices}

\question[1] What should the boolean expression be to return the maximum of the two values passed in?
\begin{lstlisting}
public int max( int a, int b ) {
	if ( _____________ ) {
		return b;
	} else {
		return a;
	}
}	
\end{lstlisting}
\begin{oneparchoices}
	\choice {\tt a < b} \choice {\tt b < a} \choice {\tt b == a} \choice {\tt b != a}
\end{oneparchoices}

\question[1] Which for loop statement could be inserted to reverse the characters in a string?
\begin{lstlisting} 
public static String reverse( String x ) {
	char[] y = new char[x.length()];
	______________________________________ {
		y[i] = x.charAt( x.length - i );
	}
	return new String(y);
}	
\end{lstlisting}
\begin{choices}
	\choice {\tt for( int i = x.length(); i > 0; i-- )}
	\choice {\tt for( int i = 0; i < x.length; i++ )}
	\choice {\tt for( int i = x.length() - 1; i >= 0; i-- )}
	\choice {\tt for( int i = 0; i < x.length() - 1; i++ )}
	\choice None of the above will work
\end{choices}

\question[1] Which of the following is true of the {\tt String} class?
\begin{choices}
	\choice Strings are immutable
	\choice Strings are mutable
\end{choices}

\end{questions}

\end{document}

