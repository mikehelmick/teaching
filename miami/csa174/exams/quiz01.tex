\documentclass[11pt,answers]{exam}
\usepackage{listings}
\usepackage{pdfsync}

%
%  Created by Mike Helmick on 2006-09-13.
%  Copyright (c) 2006 Mike Helmick. All rights reserved.
%
%

\newif\ifpdf
\ifx\pdfoutput\undefined
\pdffalse % we are not running PDFLaTeX
\else
\pdfoutput=1 % we are running PDFLaTeX
\pdftrue
\fi

\ifpdf
\usepackage{subfigure}
\usepackage[pdftex]{graphicx}
\else
\usepackage{graphicx}
\fi

% exam settings
%\boxedpoints
%\pointsinmargin
\printanswers 
%\noprintanswers

\usepackage{color} 
\definecolor{SolutionColor}{rgb}{0.8,0.9,0.9} 
\shadedsolutions 


%
%  Update these values for running headers
%
\firstpageheader{\bf\Large CSA174}{\bf\Large Quiz 01}{\bf\Large
  2007-10-22 }
\runningheader{CSA 174}{Miami University}{Quiz 01}
\addpoints

\begin{document}

\begin{center} 
  \fbox{\fbox{\parbox{5.5in}{\centering
    CSA174 - Fall 2007 - Quiz 01 \newline
	Miami University \newline
	\newline
 	There are \numquestions\  questions for a total of  \numpoints\ points.
}}}
\end{center} 

% setup standard options for the including code fragments
\lstset{language=Java,numbers=left, numberstyle=\tiny, stepnumber=1, numbersep=5pt, showstringspaces=true}

\vspace{0.1in} 
\hbox to \textwidth{Name:\enspace\hrulefill} 
Please circle your section : Section A - 10:00am Lab, Section C - 12:00pm Lab, Section D - 01:00pm Lab

\begin{center} 
\gradetable[h][questions] 
\end{center}

% Questions start here:
\begin{questions}

\section*{Multiple Choice and True/False}

\question{Circle the {\bf best} response.}
\begin{parts}
	
\part[2] Which one of these is not a Java loop keyword:
\begin{choices}
  \choice {\tt do} \choice {\tt while} \choice {\tt for} \choice {\tt if}	
\end{choices}
\begin{solution} D - if \end{solution}

\part[2] True or false, a {\tt while} loop always executes at least once:
\begin{choices}
 \choice true \choice false 
\end{choices}
\begin{solution} B - false \end{solution}
	
\part[2] A loop that never exits is called an
\begin{choices}
  \choice Continuous loop
  \choice Infinite loop
  \choice Never-ending loop
  \choice Crazy Loop
\end{choices}
\begin{solution} B - Infinite loop \end{solution}

\part[2] How many times does this loop execute? \newline
	{\tt for( int i = 1; i <= 100; i = i * 3 ) }
\begin{choices}
	\choice 33
	\choice 5
	\choice 1
\end{choices}
\begin{solution} B - 5 \end{solution}

\part[2] True or false, variables {\it declared } inside a loop are visible {\it outside} the loop.
\begin{choices}
 \choice true \choice false 
\end{choices}
\begin{solution} B - false \end{solution}

\end{parts}
	
\newpage

\section*{Code Analysis}

\question Examine the following code segment and answer the questions below.
\begin{lstlisting}
Scanner keyboard = new Scanner( System.in );	
int x = keyboard.nextInt();
int y = keyboard.nextInt();

for( int i = 0; i < x; i++ ) {
	System.out.print("Line " + i + ": ");
	int j = i;
	while( j < y ) {
		System.out.print( j + " " );
		j++;
	}
	System.out.println("");
}
\end{lstlisting}

\begin{parts}
	
	
\part[5] What is the output of this code if the user enters {\tt 3} (stored in $x$) and {\tt 2} (stored in $y$)?
\begin{solution}[1.0in]
	\newline
Line 0: 0 1\newline
Line 1: 1 \newline
Line 2:\newline
\end{solution}

\part[5] What is the output of this code if the user enters {\tt 1} and {\tt 2}?
\begin{solution}[1.0in]
	\newline
Line 0: 0 1\newline
\end{solution}

\part[5] What is the output of this code if the user enters {\tt 5} and {\tt 5}?
\begin{solution}[1.0in]
	\newline
Line 0: 0 1 2 3 4\newline 
Line 1: 1 2 3 4 \newline
Line 2: 2 3 4 \newline
Line 3: 3 4 \newline
Line 4: 4\newline
\end{solution}
	
\end{parts}

\newpage
\section*{Programming Question}

\question[15] Write a program that reads in one integer from the user and prints a multiplication table for these numbers (0 to the number), formatted to so that each number takes up 3 spaces, plus a space between each number. \newline
\begin{tt}
Enter a number: {\bf 5}\newline
0   0   0   0   0   0 \newline
0   1   2   3   4   5 \newline
0   2   4   6   8  10 \newline
0   3   6   9  12  15 \newline
0   4   8  12  16  20 \newline
0   5  10  15  20  25
\end{tt}

\begin{verbatim}
import java.util.Scanner;
public class Payroll {
  public static void main (String[] args) {
    Scanner key = new Scanner(System.in);
    // START ANSWER HERE
\end{verbatim}

\begin{solution}[4.5in]
\begin{verbatim}
System.out.println("Enter a number: ");
int num = keyboard.nextInt();

for( int i = 0; i <= num; i++ ) {
    for( int j = 0; j <= num; j++ ) {
        System.out.printf("%3d ", i*j );
    }
    System.out.println("");
}
\end{verbatim}
\end{solution}

\begin{verbatim}
    // END OF ANSWER
  }
}
\end{verbatim}



\end{questions}

\end{document}

