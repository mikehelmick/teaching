\documentclass[11pt]{exam}
\usepackage{listings}
\usepackage{pdfsync}

\newif\ifpdf
\ifx\pdfoutput\undefined
\pdffalse % we are not running PDFLaTeX
\else
\pdfoutput=1 % we are running PDFLaTeX
\pdftrue
\fi

\ifpdf
\usepackage{subfigure}
\usepackage[pdftex]{graphicx}
\else
\usepackage{graphicx}
\fi

%
%  Update these values for running headers
%
\firstpageheader{\bf\Large Homework 03}{CSA274 - Spring 2006 - Miami University \\ \bf\Large }{\bf\Large  January 26th, 2006 }
\runningheader{CSA 274}{Miami University}{Homework 03}
\addpoints

\begin{document}

\begin{center} 
  \fbox{\fbox{\parbox{5.5in}{\centering 
      {\bf Assigned: } Thursday, January 19, 2006 \newline
      {\bf Due: } Thursday, January 26, 2006 - At the beginning of class \newline
      \numquestions\  questions for a total of  \numpoints\ points.}}}
\end{center} 

% setup standard options for the including code fragments
\lstset{language=Python,numbers=left}

\vspace{0.1in} 
\hbox to \textwidth{Name:\enspace\hrulefill} 

% Questions start here:
\begin{questions}

\question[5] Define the terms {\it procedural abstraction} and {\it data abstraction}.
\vspace{2in}

\question[5] Explain the role of method preconditions and postconditions.
\vspace{2in}

\question[5] Fill in the following table: \newline
\begin{tabular}{|c|c|c|c|} \hline
	{\bf Property } & {\bf Actual Class} & {\bf Abstract Class} & {\bf Interface } \\ \hline \hline
	Instances (Objects) of this can be created & & & \\ \hline
	This can define instance variables and methods & & & \\ \hline
	This can define constants & & & \\ \hline
	The number of these a class can extend & & & \\ \hline
	The number of these a class can implement & & & \\ \hline
	This can extend another class & & & \\ \hline
	This can declare abstract methods & & & \\ \hline
	Variables of this type can be declared & & & \\ \hline
\end{tabular}


\end{questions}

\end{document}

