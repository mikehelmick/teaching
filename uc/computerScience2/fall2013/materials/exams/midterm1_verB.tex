\documentclass[11pt,answers]{exam}
\usepackage{listings}
\usepackage{pdfsync}

%
%  Created by Mike Helmick on 2006-09-13.
%  Copyright (c) 2006 Mike Helmick. All rights reserved.
%
%

\pdfoutput=1 % we are running PDFLaTeX


\usepackage{subfigure}
\usepackage[pdftex]{graphicx}

% exam settings
\boxedpoints
\pointsinmargin
\printanswers 
%\noprintanswers

\usepackage{color} 
\definecolor{SolutionColor}{rgb}{0.8,0.9,0.9} 
\shadedsolutions 


%
%  Update these values for running headers
%
\firstpageheader{\bf\Large CS 1022C}{\bf\Large Midterm Exam - VER B}{\bf\Large
  2013-10-10 }
\runningheader{CS 1022C-001/002}{University of Cincinnati}{Midterm - Fall 2013 - VER B}
\addpoints

\begin{document}

\begin{center} 
  \fbox{\fbox{\parbox{5.5in}{\centering
    CS 1022C-001/002 - Fall 2013 - Midterm Exam \newline
	University of Cincinnati \newline
	\newline
 	There are \numquestions\  questions for a total of  \numpoints\ points.
}}}
\end{center}

% setup standard options for the including code fragments
\lstset{language=C++,numbers=left, numberstyle=\tiny, stepnumber=1, numbersep=5pt, showstringspaces=true}

\vspace{0.1in} 
\hbox to \textwidth{Name:\enspace\hrulefill} 
Please circle your section : Section 001 - Dr. Talaga : Section 002 - Dr. Helmick

\section*{Instructions}
\begin{itemize}
\item Please read through this entire exam very carefully before starting. 
\item This exam is closed notes and closed books.
\item All work must be written on the exam pages in order to be graded. Any scrap paper used, must be the scrap paper provided during the exam period.
\item For programming questions: Please be accurate with your C++ syntax: this includes appropriate use of braces, semicolons, and the proper use of upper/lowercase letters.  
\item No electronic devices may be used during the exam: this includes (but is not limited to) calculators, phones, tablets, and computers.
\item You have 55 minutes to complete the exam.  
\end{itemize}

\section*{Good Luck!}

\begin{center} 
	\combinedgradetable[h]
\end{center}
\newpage

% Questions start here:
\begin{questions}

\section*{Multiple Choice and True/False}

\question{Circle the {\bf best} response.}
\begin{parts}
	
\part[2] The input to the compiler is called
\begin{choices}
  \choice Source File \choice Machine Code \choice Library Files \choice Executable program
\end{choices}
\begin{solution} A - source file \end{solution}

\part[2] True of False: It is best practice to initialize variables when you define them.
\begin{choices}
  \choice True \choice False
\end{choices}
\begin{solution} true \end{solution}

\part[2] When calling a function, the default type of parameter passing is:
\begin{choices}
  \choice pass-by-reference
  \choice pass-by-pointer
  \choice pass-by-value
  \choice pass-by-osmosis
\end{choices}
\begin{solution} C - pass-by-value \end{solution}
	
\part[2] True of False: A {\tt for} loop always executes at least one iteration	
\begin{choices}
  \choice True \choice False
\end{choices}
\begin{solution} false \end{solution}

\part[2] True of False: A {\tt do} loop always executes at least one iteration	
\begin{choices}
  \choice True \choice False
\end{choices}
\begin{solution} true \end{solution}
	
\part[2] When using pointers, the {\tt *} operator is used to 
\begin{choices}
	\choice access the pointer
	\choice dereference the pointer
	\choice assign the pointer
	\choice inspect the pointer
\end{choices}
\begin{solution}B - dereference the pointer\end{solution}
	
\part[2] If you attempt to access memory that isn't properly initialized, your program may result in a
\begin{choices}
	\choice logical error
	\choice array index out of bounds exception
	\choice segmentation fault
	\choice null pointer error
\end{choices}
\begin{solution}C - segmentation fault\end{solution}
	
%\part[2] True or False: For local variables of class types, the destructor is automatically called when the variable goes out of scope.
\part[2] True or False: The class destructor will automatically get called when an instance of that class, stored in a local variable,  goes out of scope.
\begin{choices}
	\choice True \choice False
\end{choices}
\begin{solution}True\end{solution}

\part[2] A constructor that has a single parameter, that is a reference to an instance of the same class, is called a:
\begin{choices}
	\choice Copy constructor
	\choice Duplication Constructor
	\choice Assignment operator
	\choice Destructor
\end{choices}
\begin{solution}A - Copy constructor\end{solution}

\end{parts}

\newpage
\section*{Short Answer}
\question{Write a brief response to each question. Please write in complete sentences, using a maximum of 2 sentences.}
\begin{parts}

\part[6] What is encapsulation? Why is it useful?
\begin{solution}[1.5in]
Encapsulation is the process of making data private and internal to a class. This allows you to control how the data is accessed and changed.
\end{solution}

\part[6] How do you discover syntax errors? How do you discover logic errors?
\begin{solution}[1.5in]
Syntax errors are discovered during the compiler, during program compilation. Logic errors are discovered when running your program or during unit testing.
\end{solution}
	
\end{parts}
	
\newpage

\section*{Code Analysis}

\question Examine the following code segment and answer the questions below.
\begin{lstlisting}
#include <iostream>
using namespace std;

// todo(helmick): Document this function!
int sillyFunction(int x, int &y, const int &z) {
  x = z;
  y = 50;
}

int main() {
  int x = 40;
  int y = 60;
  int z;
  cout << "x: " << x << " y: " << y << " z: " << z << endl;
  z = y;
  sillyFunction(z, y, x);
  cout << "x: " << x << " y: " << y << " z: " << z << endl;
  return 0;
}
\end{lstlisting}

\begin{parts}
	
\part[5] What is the output of this code?
\begin{solution}[1.0in]
\begin{tt}
x: 40 y: 60 z: <randomvalue>\newline
x: 40 y: 50 z: 60\newline
\end{tt}
\end{solution}

\part[5] What is the type of parameter passing used for {\tt x}, {\tt y}, and {\tt z} in {\tt sillyFunction}? Which, if any, parameters will have their changes reflected in the calling code?
\begin{solution}[1.0in]
x is pass-by-value, y is pass-by-reference, z is a const reference.\newline
the second parameter, y, can be changed since it is pass-by-reference.
\end{solution}

\part[5] What is missing from {\tt sillyFunction}?
\begin{solution}[1.0in]
It should have a return statement, but it doesn't. That is too bad.
\end{solution}
\end{parts}

\newpage
\question Examine the following code segment and answer the questions below.
\begin{parts}
\begin{lstlisting}
#include <iostream>
using namespace std;

int main() {
  int* a = new int;
  *a = 45;
  int* b = a;
  cout << "a=" << *a << " b=" << *b << endl;
  *b = 55;
  cout << "a=" << *a << " b=" << *b << endl;
  b = new int;
  *b = 65;
  cout << "a=" << *a << " b=" << *b << endl;
  *a = 75;
  cout << "a=" << *a << " b=" << *b << endl;
  delete a;
  a = NULL;
  b = NULL;
  return 0;
}
\end{lstlisting}

\part[10] What is the output of this code?
\begin{solution}[1.0in]
\begin{tt}
a=45 b=45\newline
a=55 b=55\newline
a=55 b=65\newline
a=75 b=65\newline
\end{tt}
\end{solution}

\part[5] Does this code correctly clean up all of the memory it dynamically allocates? If not, how would you fix it?
\begin{solution}[1.0in]
\begin{tt}
b is not cleaned up, it should be deleted instead of just set to NULL.
\end{tt}
\end{solution}
\end{parts}

\newpage
\section*{Programming Questions}

% This question should be common.
\question[20] Design and write the header file for a class that represents a single playing card. Playing cards are immutable, so there don't need to be any methods to change the card.
\begin{verbatim}
#ifndef CARD_H
#define CARD_H

// Assume that anything you need is included.
using namespace std;

class Card {
\end{verbatim}

\begin{solution}[4.5in]
\begin{verbatim}
public:
  Card(int suit, int rank);
  int getSuit() const;
  int getRank() const;
private:
  const int suit;
  const int rank;
  // It doesn't make sense to construct a card without suit and rank,
  // make defualt constructor private.
  Card();
\end{verbatim}
\end{solution}

\begin{verbatim}
};
\end{verbatim}

\newpage

\question[20] Write a program that reads in one integer from the user and prints a multiplication table from 0 to the number, formatted so that each number takes up 3 spaces, plus a space between each number. \newline
\begin{verbatim}
Enter a number: 5
  0   0   0   0   0   0 
  0   1   2   3   4   5 
  0   2   4   6   8  10 
  0   3   6   9  12  15 
  0   4   8  12  16  20 
  0   5  10  15  20  25
\end{verbatim}

\begin{verbatim}
#include <iostream>
using namespace std;
int main() {
  cout.width(3);  // This sets the next number printed to be of width 3.  Does not stick!
  cout << "Enter the dimension of the multiplication table? "; 
  // START ANSWER HERE
\end{verbatim}

\begin{solution}[4.5in]
\begin{verbatim}
  int dim = 0;
  cin >> dim;
  for (int y = 0; y <= dim; y++) {
    for (int x = 0; x <= dim; x++) {
      cout.width(3);
      cout << (y * x);
      cout << ' ';
    }
    cout << endl;
  }
  return 0;
\end{verbatim}
\end{solution}

\begin{verbatim}
    // END OF ANSWER
}
\end{verbatim}


\newpage
\bonusquestion[15] Implement a class called {\tt SodaCan} with functions {\tt getSurfaceArea()} and {\tt getVolume()}. In the constructor, supply the height and radius of the can. You may assume that the constant {\tt PI} (type double) is defined and available for you to use. Formulas for your reference:
\begin{center} 
  $V = \pi r^2h$ {\bf and} $A = 2 \pi r^2 + 2 \pi r h$
\end{center}

\begin{solution}
\begin{lstlisting}
class SodaCan {
public:
  SodaCan(double height, double radius) {
    this->height = height;
    this->radius = radius;
  }

  double getSurfaceArea() {
    return 2 * PI * radius*radius + 2 * PI * radius * height;	
  }

  double getVolume() {
	return PI * r * r * h;
  }

private: 
  double height;
  double radius;

  SodaCan() {};
};
\end{lstlisting}
\end{solution}

\end{questions}

\end{document}

