\documentclass[11pt,answers]{exam}
\usepackage{listings}
\usepackage{pdfsync}

%
%  Created by Mike Helmick on 2006-09-13.
%  Copyright (c) 2006 Mike Helmick. All rights reserved.
%
%

\pdfoutput=1 % we are running PDFLaTeX


\usepackage{subfigure}
\usepackage[pdftex]{graphicx}

% exam settings
\boxedpoints
\pointsinmargin
\printanswers 
%\noprintanswers

\usepackage{color} 
\definecolor{SolutionColor}{rgb}{0.8,0.9,0.9} 
\shadedsolutions 


%
%  Update these values for running headers
%
\firstpageheader{\bf\Large CS 1022C-001 (CS2)}{\bf\Large Exam 2}{\bf\Large
  2014-03-14 }
\runningheader{CS 1022C-001 (CS2)}{University of Cincinnati}{Exam 2 - Spring 2014}
\addpoints

\begin{document}

\begin{center} 
  \fbox{\fbox{\parbox{5.5in}{\centering
    CS 1022C-001 - Spring 2014 - Exam 2 \newline
	University of Cincinnati \newline
	\newline
 	There are \numquestions\  questions for a total of  \numpoints\ points. \newline
  Your total score will be computed out of 100.
}}}
\end{center}

% setup standard options for the including code fragments
\lstset{language=C++,numbers=left, numberstyle=\tiny, stepnumber=1, numbersep=5pt, showstringspaces=true}

\vspace{0.1in} 
\hbox to \textwidth{Name:\enspace\hrulefill}

\section*{Instructions}
\begin{itemize}
\item Please read through this entire exam very carefully before starting. 
\item This exam is closed book
\begin{itemize}
  \item You may use a single piece of standard 8.5inch x 11inch US letter paper. You may write whatever you like on all 6 sides of the paper, as long it contains your name somewhere. You must turn this paper in with your exam.
\end{itemize}
\item All work must be written on the exam pages in order to be graded. Any scrap paper used, must be the scrap paper provided during the exam period.
\item For programming questions: Please be accurate with your C++ syntax: this includes appropriate use of braces, semicolons, and the proper use of upper/lowercase letters.  
\item No electronic devices may be used during the exam: this includes (but is not limited to) calculators, phones, tablets, and computers.
\item You have 55 minutes to complete the exam.  
\end{itemize}

\begin{center}
{\Huge DON'T PANIC!}
\end{center}

\begin{center} 
	%\combinedgradetable[h]
  \gradetable[h]
\end{center}
\newpage

% Questions start here:
\begin{questions}

\section*{Multiple Choice and True/False}

\question{Circle the {\bf best} response.}
\begin{parts}

\part[2] Which of these data types is the largest on a 32-bit architecture?
\begin{choices}
  \choice {\tt int}
  \choice {\tt float}
  \choice {\tt int*}
  \choice {\tt they are all the same size}
\end{choices}
\begin{solution} D - {\tt they are all 32 bits on a 32 bit architecture} \end{solution}

\part[2] If a method ({\tt foo}) is declared {\tt virtual} in class {\tt A} and class {\tt B} is a subclass of {\tt A}, then the {\tt foo} method is automatically virtual in class {\tt B}.
\begin{choices} \choice true \choice false \end{choices}
\begin{solution} A - true\end{solution}

\part[2] The object on which a member function is applied is the implicit parameter named:
\begin{choices} \choice {\tt this} \choice {\tt self} \choice {\tt that} \choice {\tt object} \end{choices}
\begin{solution} A - {\tt this} \end{solution}

\part[2] Changing the implementation of a method in a derived class is called
\begin{choices}
  \choice overloading \choice extending \choice overloading \choice overdoing
\end{choices}
\begin{solution} C - overloading \end{solution}
  
\part[2] Using the arrow operator, {\tt ->}, to call a method like {\tt obj->method()} is equivalent to: 
\begin{choices}
  \choice {\tt obj*method()}
  \choice {\tt obj..method()}
  \choice {\tt *(obj.method())}
  \choice {\tt (*obj).method()}
\end{choices}
\begin{solution} D \end{solution}

\end{parts}

\newpage
\section*{Fill in the blank}

\question Fill in the blank with the best answer

\begin{parts}
\part[4] If the class {\tt B} extends the class {\tt A}, and an instance of {\tt B} is created, the constructors are called in this order:  \makebox[1in]{\hrulefill} then \makebox[1in]{\hrulefill}.
\begin{solution}[.75in] A then B \end{solution}

\part[4] If the class {\tt B} extends the class {\tt A}, and an instance of {\tt B} is destroyed, the destructors are called in this order:  \makebox[1in]{\hrulefill} then \makebox[1in]{\hrulefill}.
\begin{solution}[.75in] B then A \end{solution} 

\part[4] The termination condition of a recursive function is called the \makebox[1.5in]{\hrulefill}.
\begin{solution}[.75in] base case\end{solution}

\end{parts}

\newpage
\section*{Short Answer}

\question Provide a short answer to the following questions, using complete sentences.

\begin{parts}
\part[5] What is encapsulation? Why is it useful?
\begin{solution}[2.5in]
  Encapsulation is the hiding of data to ensure that access to it and modification of it 
\end{solution}

\part[5] What happens if you forget to delete an object that you obtained from the heap? What happens if you delete it twice?
\begin{solution}[2.5in]
  If you forget to delete - memory leak.
  If you delete it twice - double free / crash
\end{solution}
\end{parts}

\newpage

\section*{Code Analysis}

\question{Use the code listed below for all parts of this question. In this scenario each }
\begin{parts}
\begin{lstlisting}
// includes omitted to save space
class Folder; // forward declaration of Folder so it compiles
class Document {
public:
  Document(string name, int id, Folder* folder)
      : name(name), idNumber(id), folder(folder) {}
  virtual void setFolder(Folder* folder) {
    this->folder = folder;
  }
  virtual void print() {
    cout << "Document id: " << idNumber << " name: " << name << endl;
  }
protected:
  string name;
  int idNumber;
  Folder* folder; // parent folder, can be NULL
};

class Folder : public Document {
public:
  Folder(string name, int id, Folder* parent) : Document(name, id, parent) {} 
  virtual void print() {
    cout << "Folder: id: " << idNumber << " name: " << name << endl;
    for (int i = 0; i < documents.size(); i++) {
      documents.at(i)->print();
    }
  }
  vector<Document*> getDocuments() {
    return documents;
  }
  void addDocument(Document* doc) {
    doc->setFolder(this);
    documents.push_back(doc);
  }
private:
  vector<Document*> documents;
};
\end{lstlisting}
\newpage
\part[4] A variable of type {\tt Document*} can be assigned instances of type
\begin{choices}
  \choice {\tt Document*} \choice {\tt Folder*} \choice none of the above \choice all of the above  
\end{choices}
\begin{solution} D - all of the above \end{solution}

\part[4] A variable of type {\tt Folder*} can be assigned instances of type
\begin{choices}
  \choice {\tt Folder*} \choice {\tt Document*} \choice none of the above \choice all of the above  
\end{choices}
\begin{solution} A - {\tt Folder*} only \end{solution}

\part[6] Does the following code compile using only the code provided above?
\begin{lstlisting}
  Document d("input.txt", 1, NULL);
  Folder f("myDir", 2, NULL);
  f = d;
\end{lstlisting}
\begin{choices}
  \choice {\tt yes} \choice {\tt no}
\end{choices} \begin{solution} no - they cannot be assigned since not pointers.\end{solution}

\part[6] What is the output of this code, using the classes defined above. {\it This code does compile and run.}
\begin{lstlisting}
  Document d("input.txt", 1, NULL);
  Folder f("myDir", 2, NULL);
  vector<Document> v;
  v.push_back(d);
  v.push_back(f);
  for (unsigned int i = 0; i < v.size(); i++) {
    v[i].print();
  }
\end{lstlisting}
\begin{solution}[.75in]
\begin{verbatim}
Document id: 1 name: input.txt
Document id: 2 name: myDir
\end{verbatim}
  (All or nothing.)
\end{solution}
\newpage

\part[6] What is the output of this code, using the classes defined above?
\begin{lstlisting}
  Document* d = new Document("input.txt", 1, NULL);
  Folder* f = new Folder("myDir", 2, NULL);

  vector<Document*> v;
  v.push_back(d);
  v.push_back(f);
  for (unsigned int i = 0; i < v.size(); i++) {
    v[i]->print();
  }
\end{lstlisting}
\begin{solution}[1.5in]
\begin{verbatim}
Document id: 1 name: input.txt
Folder: id: 2 name: myDir
\end{verbatim}
  (All or nothing.)
\end{solution}

\part[10] What is the output of the following code, using the classes defined above?
\begin{lstlisting}
Folder* root = new Folder("/", 1, NULL);
Folder* home = new Folder("home", 2, root);
root->addDocument(home);
Folder* user = new Folder("student", 3, home);
home->addDocument(user);
Document* resume = new Document("resume.txt", 4, user);
Document* prog = new Document("program.cpp", 5, user);
user->addDocument(resume);
user->addDocument(prog);
    
root->print();  
\end{lstlisting}
\begin{solution}
\begin{verbatim}
Folder: id: 1 name: /
Folder: id: 2 name: home
Folder: id: 3 name: student
Document id: 4 name: resume.txt
Document id: 5 name: program.cp  
\end{verbatim}
\end{solution}

\end{parts}

\newpage
\question Answer the following questions using the code listing provided below.
\begin{lstlisting}
long calculate( long i, long j ) {
  if ( j == 1 ) {
    return i;
  } else {
    return i * calculate( i, j - 1 );
  }
}
   
long calculate( long i ) {
  return calculate( i, i );
}
\end{lstlisting}

\begin{parts}
   \part[3] What is the result when {\tt calculate(3) } is called?
   \begin{solution}[1in] 27 \end{solution}
   \part[3] What is the result when {\tt calculate(5) } is called?
   \begin{solution}[1in] 3125 \end{solution}
   \part[6] Describe what this code calculates:
   \begin{solution}
     {\tt calculate(i)} calculations $i$ times itself $i$ times.
   \end{solution}
\end{parts}

\newpage
\section*{Programming Questions}

\question[15] Write a recursive method to perform binary search on an array. The incoming array will already be sorted and the method should return $-1$ in the event that the target is not found in the array. Recall that binary search works by comparing the middle position to the {\tt target} value and narrowing the search space if {\tt target} is higher/lower than the value in the middle.

\begin{verbatim}
int binarySearch(int arr[], int size, int target) {
  return binarySearch(arr, target, 0, size - 1);
}

int binarySearch(int arr[], int target, int first, int last) {
  // BEGIN ANSWER
\end{verbatim}

\begin{solution}[4in]
\begin{verbatim}
  if (first > last) {
    return -1;
  }
  int mid = (last - first) / 2 + first;
  if (arr[mid] == target) {
    return mid;
  } else if (target < arr[mid]) {
    return binarySearch(arr, target, first, mid - 1);
  } else {
    return binarySearch(arr, target, mid + 1, last);
  }
\end{verbatim}
\end{solution}

\begin{verbatim}
  // END ANSWER
}
\end{verbatim}

\newpage
\question[20] This is two recursive problems in one. Write the functions necessary to do a "substring reverse" inside a string.
For example, given the string \newline
{\tt  "this is super swish."} \newline
and the "substring" to review of {\tt "is"}, the result would be:\newline
{\tt  "thsi si super swsih."}
\par
The way this works, is the substring passed in, {\tt "is"} in this case, is found in the string passed in. Each occurrence of the substring found is reversed inside the string.
\par
Recommendation: write a recursive function to partially reverse a string based on a starting and ending index. Write a recursive function to look for substrings (and trigger the reverse function if found). Write a bootstrap function if necessary. Here is the sample main code, note that the string {\tt str} is reversed in place (i.e. pass-by-reference).
\begin{lstlisting}
string str = "this is super swish.";
string sub = "is";
subReverse(str, sub);
cout << "SubReversed: " << str << endl;  
\end{lstlisting}

\begin{solution}
\begin{lstlisting}
void reverseStr(string& str, int s, int e) {
  if (s >= e) {
    return;
  }
  char tmp = str[e];
  str[e] = str[s];
  str[s] = tmp;
  reverseStr(str, s + 1, e - 1);
}
void subReverse(string &str, const string &sub, int start) {
  if (str.size() - start < sub.size()) {
    return;
  }
  bool reverse = true;
  for (int i = 0; i < sub.size(); i++) {
    if (str[start + i] != sub[i]) {
      reverse = false;
      break;
    }    
  }
  if (reverse) {
    reverseStr(str, start, start + sub.size() - 1);
  }
  subReverse(str, sub, start + 1);
}
void subReverse(string &str, const string &sub) {
  subReverse(str, sub, 0); }  
\end{lstlisting}
\end{solution}

\newpage
\begin{center}{\it continued space for question 7.}\end{center}

% In the privat above, double height = height; ?

\end{questions}

\end{document}

