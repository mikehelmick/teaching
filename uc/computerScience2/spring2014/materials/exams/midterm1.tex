\documentclass[11pt,answers]{exam}
\usepackage{listings}
\usepackage{pdfsync}

%
%  Created by Mike Helmick on 2006-09-13.
%  Copyright (c) 2006 Mike Helmick. All rights reserved.
%
%

\pdfoutput=1 % we are running PDFLaTeX


\usepackage{subfigure}
\usepackage[pdftex]{graphicx}

% exam settings
\boxedpoints
\pointsinmargin
\printanswers 
%\noprintanswers

\usepackage{color} 
\definecolor{SolutionColor}{rgb}{0.8,0.9,0.9} 
\shadedsolutions 


%
%  Update these values for running headers
%
\firstpageheader{\bf\Large CS 1022C-001 (CS2)}{\bf\Large Exam 1}{\bf\Large
  2014-02-07 }
\runningheader{CS 1022C-001 (CS2)}{University of Cincinnati}{Exam 1 - Spring 2014}
\addpoints

\begin{document}

\begin{center} 
  \fbox{\fbox{\parbox{5.5in}{\centering
    CS 1022C-001 - Spring 2014 - Exam 1 \newline
	University of Cincinnati \newline
	\newline
 	There are \numquestions\  questions for a total of  \numpoints\ points.
}}}
\end{center}

% setup standard options for the including code fragments
\lstset{language=C++,numbers=left, numberstyle=\tiny, stepnumber=1, numbersep=5pt, showstringspaces=true}

\vspace{0.1in} 
\hbox to \textwidth{Name:\enspace\hrulefill}

\section*{Instructions}
\begin{itemize}
\item Please read through this entire exam very carefully before starting. 
\item This exam is closed notes and closed books.
\item All work must be written on the exam pages in order to be graded. Any scrap paper used, must be the scrap paper provided during the exam period.
\item For programming questions: Please be accurate with your C++ syntax: this includes appropriate use of braces, semicolons, and the proper use of upper/lowercase letters.  
\item No electronic devices may be used during the exam: this includes (but is not limited to) calculators, phones, tablets, and computers.
\item You have 55 minutes to complete the exam.  
\end{itemize}

\begin{center}
{\Huge DON'T PANIC!}
\end{center}

\begin{center} 
	\combinedgradetable[h]
  %\gradetable[h]
\end{center}
\newpage

% Questions start here:
\begin{questions}

\section*{Multiple Choice and True/False}

\question{Circle the {\bf best} response.}
\begin{parts}

\part[2] What command do you use to initialize your git repository from github.uc.edu?
\begin{choices}
  \choice {\tt git clone}
  \choice {\tt git pull}
  \choice {\tt git copy}
  \choice {\tt git fetch}
\end{choices}
\begin{solution} A - {\tt git clone} \end{solution}
  
\part[2] What is the default type of parameter passing in C++?
\begin{choices}
  \choice pass-by-value
  \choice pass-by-reference
  \choice pass-by-pointer
\end{choices}
\begin{solution} A - pass-by-value \end{solution}
  
\part[2] What is value of {\tt x} and {\tt y} after these lines execute?
\begin{lstlisting}
  int x = 5;
  int y = 3 * ++x;
\end{lstlisting}
\begin{choices}
  \choice {\tt x = 5, y = 15}
  \choice {\tt x = 6, y = 16}
  \choice {\tt x = 6, y = 18}
  \choice {\tt x = 5, y = 18}
\end{choices}
\begin{solution} C - x incremented to 6 first, then multiplied by 3 for y = 18 \end{solution}
  
\part[2] A {\tt for} loop always executes at least one iteration.
\begin{choices} \choice True \choice False \end{choices}
\begin{solution} False \end{solution}
  
\part[2] All pointers in a program are the same size.
\begin{choices} \choice True \choice False \end{choices}
\begin{solution} True \end{solution}

\end{parts}

\newpage
\section*{Fill in the blank}

\question Fill in the blank with the best answer

\begin{parts}
  \part[3] The \makebox[1.25in]{\hrulefill} member function is used to find out how many characters are in a C++ string object.
  \begin{solution}[0.25in] {\tt length()} \end{solution}

  \part[3] Arrays in C++ have a starting index of \makebox[1.25in]{\hrulefill}.
  \begin{solution}[0.25in] {\tt 0} \end{solution}

  \part[3] With this function signature {\tt void myFun(int \&x)}, {\tt x} uses what type of parameter passing: \makebox[1.25in]{\hrulefill}.
  \begin{solution}[0.25in] {\tt pass-by-reference} \end{solution}
    
  \part[3] Defining multiple functions of the same name is called \makebox[1.25in]{\hrulefill}.
  \begin{solution}[0.25in] overloading \end{solution}
    
  \part[3] \makebox[1.25in]{\hrulefill} are special methods, that are used to initialize the state of an objects.
  \begin{solution}[0.25in] constructors \end{solution}
    
  \part[3] \makebox[1.25in]{\hrulefill} fields and methods on a class can be accessed anywhere you have access to an object.
  \begin{solution}[0.25in] public \end{solution}
  
  \part[3] \makebox[1.25in]{\hrulefill} is the act of hiding implementation details.
  \begin{solution}[0.25in] encapsulation \end{solution}
  
\end{parts}


\newpage

\section*{Code Analysis}

\question[15] Examine the following C++ code and identify the 5 syntax errors contained in the code
\begin{lstlisting}
#include iostream

int main() {
  cout << "Enter a character and a number";
  cin >> ch, num;
  if (ch == num);
    cout << "The character entered matches the ascii code entered." << end;
}
\end{lstlisting}
\begin{solution}
  \begin{enumerate}
   \item missing angle brackets on line 1
   \item missing namespace declaration, using namespace std
   \item variables on line 5 are not declared
   \item , should be $<<$ on line 5 
   \item should be endl on line 7
  \end{enumerate}
  
  The extra ; on 6 is not technically a syntax error, but rather a logic error.
\end{solution}

\newpage
\section*{Programming Questions}

\question[12] Leap years are years that are evenly divisible by 4, unless it is divisible by 100, unless it is also divisible by 400. For example 1900 was not a leap year, and 2000 was a leap year. Write a function called {\tt isLeap} that accepts an integer and returns {\tt true} if that number is a leap year and {\tt false} if it is not.

\begin{solution}
 \begin{lstlisting}
bool leap(int year) {
  return year % 4 == 0 && (year % 100 != 0 || year % 400 == 0);
}   
 \end{lstlisting}
\end{solution}

\newpage
\question[12] A liter is 0.264179 gallons. Write a function that takes in the number of liters of gasoline put into a car, and the number of miles that car was driven. The function will return the {\tt miles per gallon} calculation. This often happens when visiting Canada, where gasoline is sold by the liter instead of the gallon.

\begin{solution}
\begin{lstlisting}
double mpg(double liters, double miles) {
  return miles / (liters * 0.264179);
}
\end{lstlisting}  
\end{solution}

\newpage
\question[30] Write a class, called {\tt Pixel} that represents a pixel as 3 integer values for red, green, and blue. The minimum allowed value for a single color is $0$ and the maximum value allowed for any individual color is $255$. Your constructor should take in integer values for red, green, and blue and make they they are constrained to the 0 to 255 range. If the value is $< 0$, it should be made $0$ and if it is $> 255$, it should be made $255$. Your class must contain the following methods.
\begin{itemize}
  \item {\tt int getRed()}
  \item {\tt int getBlue()}
  \item {\tt int getGreen()}
  \item {\tt void brighten(double percent)}
  \item {\tt void darken(double percent)}
\end{itemize}
The {\tt brighten} method should increase the value of each color by the percent passed in, where $0.50$ represents $50\%$ increase. The $darken$ method subtracts that amount. The result of the brighten/darken operation must be made to stay within 0 and 255, so that the pixel is still valid.
\bonusquestion[15] Add a method called asHex that returns the hex representation of the RGB values with 2 digits for example, if we were to declare a pixel object like this {\tt Pixel p(122, 4, 250);}, the {\tt asHex()} method would return a string containing {\tt 0x7a04fa}. Note that the green value, $4$, is zero padded in the hex output. The leading {\tt 0x} must be added by your code as well.

\begin{solution}
\begin{lstlisting}
class Pixel {
public:
  Pixel(int nr, int ng, int nb) {
    r = normal(nr);
    g = normal(ng);
    b = normal(nb);
  }
  
  int getRed() {
    return r;
  }
  
  int getGreen() {
    return g;
  }
  
  int getBlue() {
    return b;
  }
  
  void brighten(double percent) {
    r = brighten(r, percent);
    g = brighten(g, percent);
    b = brighten(b, percent);
  }
  
  void darken(double percent) {
    r = darken(r, percent);
    g = darken(g, percent);
    b = darken(b, percent);
  }

  string asHex() {
    string s = "0x" + asString(r) + asString(g) + asString(b);
    return s;
  }

private:
  int r;
  int g;
  int b;

  static int darken(double val, double percent) {
    return normal((int) (val - val * percent));
  }
    
  static int brighten(double val, double percent) {
    return normal((int) (val + val * percent));
  }
  
  static string asString(int val) {
    char buffer[2];
    sprintf(buffer, "%2x", val);
    string s(buffer);
    return s;
  }

  static int normal(int val) {
    if (val < 0) {
      return 0;
    } else if (val > 255) {
      return 255;
    }
    return val;
  }
};  
\end{lstlisting}
\end{solution}

\newpage
\begin{center}{\tt Continued space for questions 6 and 7} \end{center}

\newpage
\begin{center}{\tt Continued space for questions 6 and 7} \end{center}

\newpage
\begin{center}{\tt SCRATCH PAPER} \end{center}

% In the privat above, double height = height; ?

\end{questions}

\end{document}

